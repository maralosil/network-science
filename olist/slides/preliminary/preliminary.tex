\documentclass[aspectratio=169]{beamer}
\usepackage{longtable}
\usepackage{xcolor}
\usepackage[T1]{fontenc}
\usepackage[english]{babel}
\usepackage[utf8]{inputenc}
\usepackage{cite}
\usepackage{amsmath}
\usepackage{amstext}
\usepackage{amssymb}
\usepackage{graphicx}
%\usepackage{psfrag}
\usepackage{tabularx}
%\usepackage{longtable}
\usepackage{xcolor}

\usepackage{caption}
\usepackage{hyperref}

\setbeamertemplate{itemize items}[square]
\setbeamertemplate{caption}[numbered]

\title{Brazilian E-Commerce Public Dataset by Olist: \\
    Insights from a Network Science Perspective}
\author{Marcelo Silva \and Rodrigo Bifulco}
%\institute{State University of Campinas}
%\date{\today}
\date{October 10, 2022}

\begin{document}

\begin{frame}
\titlepage
\end{frame}

\begin{frame}{Olist Store}
\begin{itemize}
    \item Brazilian startup that operates in the e-commerce segment, mainly
            through the marketplace \cite{olist-wiki}
    \item Concentrates sellers who want to advertise on marketplaces such as:
    \begin{itemize}
        \item Mercado Livre
        \item B2W
        \item Via Varejo
        \item Amazon
    \end{itemize}
    \item Concentrates the products of all sellers in a single store that is
            visible to the end consumer
\end{itemize}
\end{frame}

\begin{frame}{Dataset}
\begin{itemize}
    \item Brazilian ecommerce public dataset of orders made at Olist Store
        \cite{olist-sionek}
    \item Information of 100k orders from 2016 to 2018 made at multiple
            marketplaces
    \item Allows viewing an order from multiple dimensions such as: 
    \begin{itemize}
        \item order status
        \item price
        \item payment
        \item freight performance to customer location
        \item product attributes 
        \item reviews written by customers
    \end{itemize}
\end{itemize}
\end{frame}

\begin{frame}{Network}
\begin{itemize}
    \item The network dataset has been created from the merge of some of the
            datasets made available by Olist and also by population data from
            IBGE.
    \item Nodes are order ids, states and product category
    \item Links are connections between:
        \begin{itemize}
            \item order ids and states
            \item states and product category
        \end{itemize}
    \item Size is around 316872 nodes and 205138 links
    \item The network is tripartite
\end{itemize}
\end{frame}

\begin{frame}{Questions}
\begin{itemize}
    \item Which state and product category generated the highest average profit
            for the marketplace between 2016 and 2018?
    \item Which state and product category generated the lowest average profit
            for this marketplace between 2016 and 2018?
    \item Is the profit proportional to the population of the state?
\end{itemize}
\end{frame}

\begin{frame}{References}
\bibliographystyle{plain}

\begin{thebibliography}{2}

\bibitem{olist-wiki} Olist. [Online]. \url{https://pt.wikipedia.org/wiki/Olist}.
(Accessed Oct 4, 2022).

\bibitem{olist-sionek} Olist and André Sionek. (2018). {\em Brazilian
E-Commerce Public Dataset by Olist} [Data set]. Kaggle.
\url{https://doi.org/10.34740/KAGGLE/DSV/195341}

\end{thebibliography}
\end{frame}
\end{document}
